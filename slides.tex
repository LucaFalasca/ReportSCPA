\documentclass[compress]{beamer}
\usepackage[
    title={Parallel Matrix Multiplication},
    event={Progetto fine corso SCPA},
    author={MC, LF},
    longauthor={Matteo Conti, Luca Falasca},
    email={},
    institute={SCPA 2023-2024},
    longinstitute={Universita' degli Studi di Roma Tor Vergata},
]{unislides}
\usepackage{graphicx} % Required for inserting images
\usepackage{minted}
\usepackage{algorithm}
\usepackage{hyperref}
\usepackage{adjustbox}
\usepackage{svg}
\svgsetup{inkscapelatex=false}

\begin{document}

\begin{frame}[plain]
    \titlepage
\end{frame}

%---------------------INTRODUZIONE---------------------
\section{Introduzione}

\subsection{Descrizione del problema}
\begin{frame}{\secname \text{ }- \subsecname\ }
    %TODO
\end{frame}

\subsection{Obiettivi}
\begin{frame}{\secname \text{ }- \subsecname\ }
    %TODO
\end{frame}

\subsection{Metriche di valutazione}
\begin{frame}{\secname \text{ }- \subsecname\ }
    %TODO
\end{frame}

\subsection{Raccolta dei dati}
\begin{frame}{\secname \text{ }- \subsecname\ }
    %TODO
\end{frame}

%---------------------MPI------------------------------
\section{MPI}

\begin{frame}{\secname}
    %TODO
\end{frame}

\subsection{Distribuzione del carico}
\begin{frame}{\secname \text{ }- \subsecname\ }
    %TODO
\end{frame}

\subsection{Riduzione del risultato}
\begin{frame}{\secname \text{ }- \subsecname\ }
    %TODO
\end{frame}

\subsection{Implementazione del prodotto}
\begin{frame}{\secname \text{ }- \subsecname\ }
    %TODO
\end{frame}

\subsubsection*{Implementazione Naive}
\begin{frame}{\secname \text{ }- \subsecname\ \text{ }- \subsubsecname}
    %TODO
\end{frame}

\subsubsection*{Implementazione Column blocked}
\begin{frame}{\secname \text{ }- \subsecname\ \text{ }- \subsubsecname}
    %TODO
\end{frame}

%---------------------CUDA------------------------------
\section{CUDA}

\begin{frame}{\secname}
    %TODO
\end{frame}

\subsection{1 versione}
\begin{frame}{\secname \text{ }- \subsecname\ }
    %TODO
\end{frame}

\subsection{2 versione}
\begin{frame}{\secname \text{ }- \subsecname\ }
    %TODO
\end{frame}

\subsection{3 versione}
\begin{frame}{\secname \text{ }- \subsecname\ }
    %TODO
\end{frame}

\subsection{Configurazione dei parametri}
\subsubsection*{Thread}
\begin{frame}{\secname \text{ }- \subsecname\ \text{ }- \subsubsecname}
    %TODO
\end{frame}

\subsubsection*{Shared memory}
\begin{frame}{\secname \text{ }- \subsecname\ \text{ }- \subsubsecname}
    %TODO
\end{frame}

\subsubsection*{Bank conflit}
\begin{frame}{\secname \text{ }- \subsecname\ \text{ }- \subsubsecname}
    %TODO
\end{frame}

%---------------------MPI+CUDA------------------------------
\section{MPI+CUDA}

\begin{frame}{\secname}
    %TODO
\end{frame}

%---------------------Analisi delle prestazioni------------------------------
\section{Analisi delle prestazioni}

\subsection{MPI}
\begin{frame}{\secname \text{ }- \subsecname\ }
    %TODO
\end{frame}

\subsubsection*{Matrici quadrate}
\begin{frame}{\secname \text{ }- \subsecname\ \text{ }- \subsubsecname}
    %TODO
\end{frame}

\subsubsection*{Matrici rettangolari}
\begin{frame}{\secname \text{ }- \subsecname\ \text{ }- \subsubsecname}
    %TODO
\end{frame}

\subsection{CUDA}
\begin{frame}{\secname \text{ }- \subsecname\ }
    %TODO
\end{frame}

\subsubsection*{Matrici quadrate}
\begin{frame}{\secname \text{ }- \subsecname\ \text{ }- \subsubsecname}
    %TODO
\end{frame}

\subsubsection*{Matrici rettangolari}
\begin{frame}{\secname \text{ }- \subsecname\ \text{ }- \subsubsecname}
    %TODO
\end{frame}

\subsection{MPI+CUDA}
\begin{frame}{\secname \text{ }- \subsecname\ }
    %TODO
\end{frame}

\subsubsection*{Matrici quadrate}
\begin{frame}{\secname \text{ }- \subsecname\ \text{ }- \subsubsecname}
    %TODO
\end{frame}

\subsubsection*{MPI+CUDA}
\begin{frame}{\secname \text{ }- \subsecname\ \text{ }- \subsubsecname}
    %TODO
\end{frame}


\begin{frame}
    \frametitle{Grazie per l'attenzione!}
    \begin{itemize}
        \item Tutto il codice che implementa il progetto è disponibile al
        seguente repository: \href{https://github.com/LucaFalasca/ParallelMatrixMultiplication}{https://github.com/LucaFalasca/ParallelMatrixMultiplication}
        \item contattaci a:
            \begin{itemize}
                \item \href{mailto:matteo.conti@students.uniroma2.eu}{matteo.conti@students.uniroma2.eu}
                \item \href{mailto:luca.falasca@students.uniroma2.eu}{luca.falasca@students.uniroma2.eu}
            \end{itemize}
       \end{itemize}
\end{frame}


\end{document}
